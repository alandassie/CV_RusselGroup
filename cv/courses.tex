%-------------------------------------------------------------------------------
% SECTION TITLE
%-------------------------------------------------------------------------------
\cvsection{Additional Courses}


%-------------------------------------------------------------------------------
% CONTENT
%-------------------------------------------------------------------------------
\begin{cventries}

%---------------------------------------------------------
  \cventry
    {FRIB-TA Summer School: A practical walk through formal scattering theory: Connecting bound states, resonances, and scattering states in exotic nuclei and beyond} % Course title
    {Michigan State University} % Organisation
    {Virtual} % Location
    {August 2021} % Date(s)
    {
      \begin{cvitems} % Description(s) of tasks/responsibilities
        \item {This summer school offered me an introduction to nonrelativistic quantum scattering theory. Formal aspects, centered around the important concept of the S-matrix, were covered in detail, complemented at each step by numerical illustrations and hands-on programming exercises.}
        \item {\textbf{Organizers and Lecturers:} Kevin Fossez (ANL), Sebastian Koenig (NCSU), Heiko Hergert (MSU).}
      \end{cvitems}
    }

%---------------------------------------------------------
  \cventry
    {Winter School: Applications of Artificial Intelligence to Topics in Nuclear Physics} % Course title
    {University of Maryland, The Catholic Univ. of America, Jefferson Lab} % Organisation
    {Virtual} % Location
    {January 2021} % Date(s)
    {
      \begin{cvitems} % Description(s) of tasks/responsibilities
        \item {The AI4NP Winter School gave me a deeper understanding on what Artificial Intelligence and Machine Learning are and how they can be used to analyze nuclear physics data and perform theoretical calculations of nuclear many-body systems. The AI4NP lecture topics emphasized in active Nuclear Physics research that relies on AI/ML techniques, as well as synergies between the computer science and the NP communities.}
        \item {\textbf{Lecturers:} Michelle Kuchera (Davidson College), Alessandro Lovato (ANL), Corey Adams (ANL), Cristiano Fanelli (MIT), Nobuo Sato (Jefferson Lab), \textbf{Organizers:} Paulo Bedaque (UMD), Amber Boehnlein (JLab), Tanja Horn (CUA).}
      \end{cvitems}
    }

%---------------------------------------------------------
  \cventry
    {Research Internship: Calibration of a FD-SOI sensor for dosimetry in low orbits.} % Course title
    {Balseiro Institute, National University of Cuyo - National Atomic Energy Commission} % Organisation
    {Bariloche, Argentina} % Location
    {February 2020} % Date(s)
    {
      \begin{cvitems} % Description(s) of tasks/responsibilities
        \item {\textbf{Supervisors} Fabricio Alcalde Bessia, José Lipovetzky}
        \item {This work consisted in the development of an embedded system to carry out experiments of characterization of the response of the sensors at low dose rates, similar to those present in the orbit of the satellite. The work included the assembly of electronic circuits, programming of an Arduino platform, learning about MOS device physics and how to perform experiments with radiation sources.}
      \end{cvitems}
    }
    
%---------------------------------------------------------
  \cventry
    {School of Applied Neutronic Techniques.} % Course title
    {Argentine Laboratory of Neutron Beams - National Atomic Energy Commission} % Organisation
    {Buenos Aires, Argentina} % Location
    {October 2019} % Date(s)
    {This was an intensive two-week school, of a theoretical-experimental nature and whose main objective was to strengthen the scientific-technological capacity of Argentina and the region in relation to the use of neutron techniques. It was aimed at professionals, technologists, advanced level undergraduate students and postgraduate students.}
        
%---------------------------------------------------------
  \cventry
    {Operator of personal computers and utility programs.} % Course title
    {National University of Asunción} % Organisation
    {Asunción, Paraguay} % Location
    {Mar 2015 - Dec 2015} % Date(s)
    {This was a university course of 840 hours, which included subjects such as algorithms, graphing, operating systems and databases.}
    
\end{cventries}
